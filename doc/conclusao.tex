\chapter{Conclusão}\label{cap:conclusao}

Este trabalho objetivou desenvolver uma ferramenta de representação
comportamental baseada em otimização para futebol de robôs, com meta
intermediária de criar um modelo discreto sequencial para o problema do futebol
de robôs.

Foi desenvolvido um modelo abstrato do futebol de robôs no
Capítulo~\ref{cap:modelagem}.  Esse modelo foi base para o programa apresentado
no Capítulo~\ref{cap:arquitetura}.  Essa ferramenta utiliza uma combinação de
gradiente da função objetivo, buscas aleatórias e posições chaves para
selecionar jogadas ótimas dentro do tempo disponível para o planejamento.

A ferramenta criada atingiu os objetivos desejados, gerando uma variedade de
comportamentos através da modificação dos parâmetros da função objetivo,
conforme mostrado no Capítulo~\ref{cap:resultados}.  Também foi desenvolvida com
sucesso uma interface gráfica que permite modificar esses parâmetros em tempo de
execução.

O trabalho atual permite que seja criado um time superior apenas modificando os
parâmetros da função objetivo ou adicionando novos custos à função objetivo.
Também é possível utilizar esse modelo para aplicar outros métodos de otimização.

Como trabalhos futuros, sugere-se desenvolver um sistema automático para
realizar as partidas e avaliar o desempenho do time nessa partida.  Isso
permitiria que algorítimos de otimização também fossem utilizados para
melhorar os parâmetros.  Para isso, é a necessário um juiz automático
para permitir jogos não supervisionados.

% vim: tw=80 et ts=2 sw=2 sts=2 ft=tex spelllang=pt_br,en
