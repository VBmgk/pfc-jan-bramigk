\chapter{Conclusão}\label{cap:conclusao}

% TODO[vbramigk] Refazer conclusão:
% - Recontextualizar
% - Listar objetivos (com linha de onde foram cumpridos)
% - Retificando o que forem concluindo
% - Apresentar as contribuições
% - Trabalhos futuros

Este trabalho objetiva desenvolver um algoritmo de controle para o futebol
de robôs que obtenha um desempenho melhor que o utilizado atualmente, de acordo
com um critério que será definido posteriormente.
Como objetivos secundários, este trabalho objetiva criar um modelo discreto para
o problema do futebol de robôs, que é um jogo contínuo de soma zero.
A partir desta discretização, pretende-se desenvolver uma arquitetura de
controle com base no algoritmo minimax para se planejar jogadas.

Foi estudada a teoria dos jogos, e, mais especificamente, o algoritmo minimax
no capítulo~\ref{cap:minimax}. Alguns dados empíricos observados com testes
observados anteriormente no capítulo~\ref{cap:licoes_aprendidas}.

Foi desenvolvido um modelo abstrato do futebol de robôs no
capítulo~\ref{cap:mapeamento}. Este modelo foi base para o programa apresentado
no capítulo~\ref{cap:programa}, apesar de não ter sido terminado ainda.

Nas próximas etapas serão implementadas as métricas para comparar os resultados
obtidos com a nova arquitetura.

É necessário concluir o programa e coletar os dados através das métricas para
que se chegue a alguma conclusão.

Como trabalho futuro, é sugerido que seja implementado preditor baseado em
um algoritmo com ANN para modelar o time adversário, por reduzir o tamanho da
arvore de planejamento, oque permitiria que mais jogadas fossem consideradas.

% vim: tw=80 et ts=2 sw=2 sts=2
