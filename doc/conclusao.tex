%\addcontentsline{toc}{chapter}{Conclusão}
\chapter{Conclusão}\label{cap:conclusao}

% TODO:
% OBS: - Recontextualizar trabalho
%      - Retificação do cumprimento dos objetivos
%      - Listagem das contribuições
%      - Trabalhos futuros
%      - Conjugar no passado

Este trabalho objetiva desenvolver uma ferramenta de representação
comportamental baseada em otimização para futebol de robôs.  A meta
intermediária é criar um modelo discreto sequencial para o problema do futebol
de robôs. A partir desta discretização, foi desenvolvida uma arquitetura de
controle que seleciona jogadas o mais próximo da jogada ótima possível, de
acordo com uma função de avaliação e dentro do tempo disponível para o
planejamento.

Foi desenvolvido um modelo abstrato do futebol de robôs no
Capítulo~\ref{cap:modelagem}. Este modelo foi base para o programa apresentado
no Capítulo~\ref{cap:arquitetura}.

A ferramenta criada atingiu os objetivos desejados, modificando o comportamento
do time através da modificação dos parâmetros da função objetivo, conforme
mostrado no Capítulo~\ref{cap:resultados}. Também foi desenvolvida com sucesso
uma interface gráfica que permite modificar esses parâmetros em tempo de
execução.

O trabalho atual pode permitir que um time melhor seja criado modificando-se os
parâmetros da função objetivo ou adicionando novos custos à função objetivo.
Através do fornecimento dos parâmetros vencedores da competição anterior, tem-se
um avanço no desempenho do time.

Pode-se melhorar o time utilizando uma maneria automática para realizar as
partidas e avaliar o desempenho do time nessa partida. Isso permitiria que
algorítimos de otimização também fossem utilizados para melhorar os parâmetros.
Para isso, é a necessário um juíz automático para permitir jogos não
supervisionados por humanos.
% TODO: Escolher outra palavra, 'humanos' parece estranho.

% vim: tw=80 et ts=2 sw=2 sts=2 ft=tex
