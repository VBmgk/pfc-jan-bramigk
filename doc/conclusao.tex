\addcontentsline{toc}{chapter}{Conclusão}
\chapter*{Conclusão}\label{cap:conclusao}

Este trabalho objetiva desenvolver uma ferramenta de representação
comportamental baseado em otimização para futebol de robôs.
Como objetivo intermediário, este trabalho objetiva criar um modelo discreto
sequencial para o problema do futebol de robôs, que é um jogo contínuo de soma
zero. A partir desta discretização, foi desenvolvida uma arquitetura de
controle que seleciona jogadas ótimas de acordo com uma função de avaliação.

Foi desenvolvido um modelo abstrato do futebol de robôs no
capítulo~\ref{cap:modelagem}. Este modelo foi base para o
programa apresentado no capítulo~\ref{cap:arquitetura}.

A ferramenta criada atingiu os objetivos desejados, modificando o
comportamento do time através da modificação dos parâmetros da função
objetivo, conforme mostrado no capítulo~\ref{cap:resultados}. Também foi
desenvolvida com sucesso uma interface gráfica que permite modificar
esses parâmetros em tempo de execução.

Como trabalho futuro, é sugerido que seja implementado 
um algorítimo de controle adaptativo para selecionar os
parâmetros ótimos durante a partida.

% vim: tw=80 et ts=2 sw=2 sts=2
