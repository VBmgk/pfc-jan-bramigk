\addcontentsline{toc}{chapter}{Conclusão}
\chapter*{Conclusão}\label{cap:conclusao}

Este trabalho objetiva desenvolver uma ferramenta de representação
comportamental baseado em otimização para futebol de robôs.
A meta intermediária é criar um modelo discreto sequencial
para o problema do futebol de robôs. A partir desta
discretização, foi desenvolvida uma arquitetura de controle
que seleciona jogadas o mais próximo da jogada ótima possível,
de acordo com uma função de avaliação e dentro do tempo disponível
para o planejamento.

Foi desenvolvido um modelo abstrato do futebol de robôs no
Capítulo~\ref{cap:modelagem}. Este modelo foi base para o
programa apresentado no Capítulo~\ref{cap:arquitetura}.

A ferramenta criada atingiu os objetivos desejados, modificando o
comportamento do time através da modificação dos parâmetros da função
objetivo, conforme mostrado no Capítulo~\ref{cap:resultados}. Também foi
desenvolvida com sucesso uma interface gráfica que permite modificar
esses parâmetros em tempo de execução.

Como trabalho futuro, é sugerido que seja implementado 
um algorítimo de controle adaptativo para selecionar os
parâmetros ótimos durante a partida.

% vim: tw=80 et ts=2 sw=2 sts=2
