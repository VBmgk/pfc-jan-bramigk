% resumo em português
\setlength{\absparsep}{18pt} % ajusta o espaçamento dos parágrafos do resumo
\begin{resumo}

  Este trabalho apresenta uma ferramenta de representação comportamental baseada
  em otimização para futebol de robôs para uma equipe da \textit{Small Size
  League} (SSL) da RoboCup, RoboIME (representando o IME nesta competição).  As
  equipes da SSL são geralmente controlados por heurísticas puras.  Isso
  restringe o movimento dos robôs para um conjunto fixo de comportamentos,
  limitando as possíveis jogadas.

  O objetivo deste trabalho é desenvolver uma ferramenta de representação
  comportamental baseada em otimização para futebol de robôs.

  Para permitir que várias jogadas sejam simuladas, o jogo é abstraído.  Isso é
  importante pois o sistema varia continuamente devido à presença de robôs da
  equipe adversária.  Assim, um jogo contínuo é abstraído em um discreto e
  sequencial, semelhante ao xadrez.  Com base nessa abstração, foi definida uma
  função utilidade, o que permitiu uma abordagem de otimização ao problema.

  Foi implementada uma arquitetura de controle com base nessa abstração.  Como
  plataforma de testes, parte do software já em uso pela RoboIME
  (\textit{pyroboime}) é usado, o que resolve o controle físico dos robôs
  através da implementação de máquinas de estado de táticas para as ações
  abstraídas.  A ferramenta é capaz de controlar um time de robôs, gerando
  comportamentos que variam entre bloquear chutes diretos, avançar para a quadra
  adversária e se posicionar para atacar diretamente através de passes.

  \textbf{Palavras-chaves}: inteligência artificial, discretização, otimização, robótica, robocup.
\end{resumo}

% resumo em inglês
\begin{resumo}[Abstract]\begin{otherlanguage*}{english}

  This paper presents a behavioral representation tool based on optimization for
  robot soccer for a RoboCup's Small Size League (SSL) team, RoboIME
  (representing IME in this competition).  SSL teams are usually controlled by
  pure heuristics.  This restricts the movement of the robots to a fixed set of
  behaviors, limiting the possible plays.

  The objective of this work is to develop a behavioral representation tool
  based on optimization for robot soccer.

  To allow several plays to be simulated, the game is abstracted.  This is
  important because the system continuously varies due to the presence of robots
  from the opposing team.  Thus, a continuous game is abstracted into a discrete
  and sequential one, like chess.  Based on this abstraction, a utility function
  was defined, allowing an optimization approach to the problem.

  A control architecture based on that abstraction was implemented.  As testing
  platform, part of the software already in use by RoboIME (pyroboime) is used,
  which solves the physical control of the robots by implementing state machines
  for the tactics of the abstracted actions.  The tool is able to control a set
  of robots, generating behaviors ranging from blocking direct kicks, advancing
  to the opponent's court and positioning for direct attacks through passes.

  \textbf{Key-words}: artificial intelligence, discretization, optimization, robotics, robocup.
\end{otherlanguage*}\end{resumo}

% vim: tw=80 et ts=2 sw=2 sts=2 ft=tex
