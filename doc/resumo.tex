% resumo em português
\setlength{\absparsep}{18pt} % ajusta o espaçamento dos parágrafos do resumo
\begin{resumo}

  % TODO; Inserir tema, objetivo e contribuição

  Este trabalho apresenta um sistema de controle de um time da small size league
  (SSL) utilizando algoritmo minimax. Os times da SSL normalmente são comandados
  por heurísticas puras. Isso restringe o movimento dos robôs a um determinado
  conjunto fixo de comportamentos, restringindo as jogadas possíveis.

  Para utilizar o algoritmo minimax é necessário abstrair o jogo de modo que se
  possa simular mais situações e permitir que várias sequências de jogadas
  possam ser consideradas em tempo real, por se tratar de um ambiente dinâmico.
  Assim, um jogo contínuo é abstraído em um discreto e sequencial,
  semelhante a um jogo de xadrez. Partindo-se desta abstração, aplica-se o
  algoritmo minimax.

  Com o objetivo de verificar a viabilidade dessa abordagem, foi implementa uma
  arquitetura de controle baseada em um modelo discreto sequencial para a
  categoria SSL da Robocup. Como plataforma de testes, será utilizada a pyroboime
  que resolve o problema de controle físico dos robôs. Como resultados futuros,
  será desenvolvido um algoritmo de planejamento superior ao utilizado
  atualmente na pyroboime.

  \textbf{Palavras-chaves}: inteligência artificial, minimax, robótica.
\end{resumo}

% resumo em inglês
\begin{resumo}[Abstract]
  \begin{otherlanguage*}{english}

    % TODO[jansegre] traduzir depois de fazer o resumo

    The following work is a partial study that aims to implement a control
    architecture on top of discrete sequential model for the Robocup SSL
    competition which the authors take part within the RoboIME team, a
    representative of the IME on that competition.

    \textbf{Key-words}: artificial intelligence, minimax, robotics.
  \end{otherlanguage*}
\end{resumo}

% vim: tw=80 et ts=2 sw=2 sts=2
