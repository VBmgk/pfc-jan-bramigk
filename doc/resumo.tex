% resumo em português
\setlength{\absparsep}{18pt} % ajusta o espaçamento dos parágrafos do resumo
\begin{resumo}

  Este trabalho apresenta uma ferramenta de representação comportamental baseado
  em otimizacao para futebol de robôs de um time da \textit{Small
  Size League} (SSL) de futebol de robôs para a equipe RoboIME (que representa o
  IME nessa competição). Os times da SSL normalmente são comandados por
  heurísticas puras. Isso restringe o movimento dos robôs a um determinado
  conjunto fixo de comportamentos, restringindo as jogadas possíveis.

  O objetivo deste trabalho é desenvolver uma ferramenta de representação
  comportamental baseado em otimização para futebol de robôs.

  Para permitir que várias jogadas sejam simuladas é necessário abstrair
  o jogo de modo que se possa simular mais situações e permitir que várias
  sequências de jogadas possam ser consideradas em tempo real. Isso é importante
  pois o sistema varia continuamente devido a presença dos robôs time adversário.
  Assim, um jogo contínuo é abstraído em um discreto e sequencial,
  semelhante a um jogo de xadrez. Partindo-se desta abstração, é construida
  uma função utilidade, permitido assim que o problema seja tratado como um problema
  de otimização.

  A arquitetura de controle baseada em um modelo discreto sequencial do jogo foi
  implementada. Como plataforma de testes, é utilizada parte da \textit{pyroboime}, que
  resolve o problema de controle físico dos robôs. Com resultados futuros, será
  desenvolvido um algoritmo de planejamento superior ao utilizado atualmente no
  \textit{pyroboime}.

  \textbf{Palavras-chaves}: inteligência artificial, discretização, otimização, robótica, robocup.
\end{resumo}

% resumo em inglês
% TODO: traduzir o resumo
\begin{resumo}[Abstract]
  \begin{otherlanguage*}{english}
     \textbf{Abstract em produção}
%    This work presents a control system for robot soccer team of the Small Size
%    League (SSL) for the RoboIME team (which stands for IME on that competition)
%    using a Minimax algorithm. The teams of SSL are usually controlled by plain
%    heuristics. That restricts the movement of the robots to a limited set of
%    behaviours, shortening the range of possible plays.
%
%    The objective of this work is to implement that system in such a way that
%    its performance is at least comparable to that of the current intelligence
%    being used by RoboIME (namely \textit{pyroboime}), which is heuristic.
%
%    For using a Minimax algorithm it is necessary to abstract the game in such a
%    way that it can reach a whole range of moves in real time using few steps.
%    In that manner a continous game is abstracted into a discrete sequential
%    one, similar to chess. Based off that abstraction the Minimax algorithm is
%    built.
%
%    With the purpose of verifying the feasability of this approach, a control
%    architecture has been implemented on top of a discrete sequential model of
%    the game. As a test platform part of \textit{pyrobime} is used, which solves
%    the issue of controlling the real robots. With future results a planning
%    algorithm will be developed superior to the one currently in use on
%    \textit{pyroboime}.
%
%    %As of this publishing the objective has not been met.
%
%    \textbf{Key-words}: artificial intelligence, minimax, robotics, robocup.
  \end{otherlanguage*}
\end{resumo}


































% vim: tw=80 et ts=2 sw=2 sts=2
