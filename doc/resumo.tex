% resumo em português
\setlength{\absparsep}{18pt} % ajusta o espaçamento dos parágrafos do resumo
\begin{resumo}

  Este trabalho apresenta uma ferramenta de representação comportamental baseada
  em otimização para futebol de robôs para uma equipe da \textit{Small Size
  League} (SSL) da RoboCup, RoboIME (representando IME nesta competição).
  Equipes da SSL são geralmente controlados por heurísticas puras. Isso
  restringe o movimento dos robôs para um conjunto fixo de comportamentos,
  limitando as possíveis jogadas.

  O objetivo deste trabalho é desenvolver uma ferramenta de representação
  comportamental baseada em otimização para futebol de robôs.

  Um modelo discreto e sequencial no domínio das ações foi criado, juntamente
  com uma função utilidade. Isso possibilitou que uma busca seja feita para
  encontrar jogadas com boa avaliação de acordo com essa função. Foi
  implementada uma arquitetura de controle baseada nessa abstração, onde é feita
  uma busca pela combinação do método de descida do gradiente aplicado a um
  conjunto de jogadas aleatórios, ações sugeridas (inseridos através da GUI) e o
  planejamento anterior. A ferramenta é capaz de controlar um conjunto de robôs,
  dando origem a comportamentos que vão desde bloquear chutes diretos, avançar
  para a quadra adversária e se posicionar para atacar através de passes.

  \textbf{Palavras-chaves}: inteligência artificial, discretização, otimização, robótica, robocup.
\end{resumo}

% resumo em inglês
\begin{resumo}[Abstract]\begin{otherlanguage*}{english}

  This paper presents a behavioral representation tool based on optimization for
  robot soccer for a RoboCup's Small Size League (SSL) team, RoboIME
  (representing IME in this competition). SSL teams are usually controlled by
  pure heuristics. This restricts the movement of the robots to a fixed set of
  behaviors, limiting the possible plays.

  The objective of this work is to develop a behavioral representation tool
  based on optimization for robot soccer.

  A discreet and sequential model in the field of actions was created, along
  with a utility function. This enabled a search to be made to find plays with
  good evaluation according to that function. A control architecture based on
  this abstraction was implemented, where a search is made by combining a
  gradient descent method applied to a set of random moves, suggested actions
  (inserted through the GUI) and the previous planning. The tool is able to
  control a set of robot, yielding behaviors which range from blocking direct
  kicks, advancing to the opponent's court and positioning for attacking through
  passes.

  \textbf{Key-words}: artificial intelligence, discretization, optimization, robotics, robocup.
\end{otherlanguage*}\end{resumo}

% vim: tw=80 et ts=2 sw=2 sts=2 ft=tex
