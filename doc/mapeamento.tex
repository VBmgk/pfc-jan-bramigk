\chapter{Mapeamento de jogo contínuo para um jogo discreto}\label{cap:mapeamento}

Uma das dificuldades de se discretizar um sistema é a necessidade de criar uma abstração válida para
o jogo, de modo que o que a ocorra na simulação acontessa na prática caso a mesma situação simulada
seja observada no mundo físico.

Essa modelagem pode ser separada em duas etapas:

\begin{itemize}
  \item Representação do jogo
  \item Execução do planejamento
\end{itemize}

\section{Representação do jogo}

Cada um dos robôs em campo será modelado com as seguintes ações possíveis:
\begin{itemize}
  \item Time com a bola
  \begin{itemize}
    \item robô sem a bola
    \begin{itemize}
       \item Move
    \end{itemize}
    \item robô com a bola
    \begin{itemize}
       \item Move
       \item Chute
       \item Passe
    \end{itemize}
  \end{itemize}
  \item Time sem a bola:
  \begin{itemize}
    \item Move
    \item GetBall
  \end{itemize}
\end{itemize}

\section{Execução do planejamento}

Esta etapa do modelo é o que decide se o planejamento será executado da maneira como foi simulado
no mundo virtual.
