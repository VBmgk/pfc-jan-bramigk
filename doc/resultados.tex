\chapter{Resultados}\label{cap:resultados}
% OBS: "Falta de diagramas"

\section{Interface Gráfica}\label{sec:gui}

\FloatBarrier

\begin{figure}[h]
  \centering
  \includegraphics[width=0.8\linewidth]{gui_field}
  \caption{Aparência geral e representação do campo}\label{fig:gui_field}
\end{figure}

A ferramenta representa o estado atual do jogo desenhando o campo, os robôs e a
bola, como pode ser visto na Figura~\ref{fig:gui_field}.  O campo não faz parte
do estado mas é necessário para ter uma referência visual das posições.

\begin{figure}[h]
  \centering
  \includegraphics[width=0.4\linewidth]{gui_zoom_out}
  \includegraphics[width=0.4\linewidth]{gui_zoom_in}
  \caption{Zoom e arraste do campo}\label{fig:gui_zoom}
\end{figure}

Para visualizar situações com mais detalhes a ferramenta permite zoom e arraste
do campo (Figura~\ref{fig:gui_zoom}).

\FloatBarrier

\begin{figure}[h]
  \centering
  \includegraphics[width=0.75\linewidth]{gui_params}
  \caption{Parâmetros configuráveis}\label{fig:gui_params}
  \includegraphics[width=0.75\linewidth]{gui_widgets}
  \caption{Controles}\label{fig:gui_widgets}
\end{figure}

\FloatBarrier

% OBS: "???" em cima das palavras em itálico
Existem quatro abas disponíveis: \textit{Main}, \textit{Calibration},
\textit{Suggestions}, \textit{Draw options}. As abas são retráteis e
translucidas para que as mudanças de configurações possam ser observadas
facilmente.  Na Figura~\ref{fig:gui_field}, por exemplo, todas as estão
minimizadas, na Figura~\ref{fig:gui_params} é mostrada a aba
\textit{Calibration}, as outras três podem ser vistas na
Figura~\ref{fig:gui_widgets}.

\begin{figure}[h]
  \centering
  \includegraphics[width=0.4\linewidth]{gui_ball_move}
  \includegraphics[width=0.4\linewidth]{gui_ball_stop}
  \caption{Representação do tempo para chegar na bola}\label{fig:gui_ball}
\end{figure}

A interface também possui indicadores para o tempo de chegar na bola, o robô que
chegará primeiro na bola e os robôs que podem receber um passe.

O indicador do robô que chegará primeiro, às vezes chamado de "dono da bola", é
representado com um círculo rosa em torno do robô.

O indicador para o tempo de chegar na bola está representado na
Figura~\ref{fig:gui_ball} como um campo de pontos translúcidos ao redor da bola,
quanto maior o diâmetro do ponto menor o tempo para se chegar na bola a partir dali.
% OBS: ele circulou o 'à', mas acho que esta certo
Na imagem à direita, a bola está parada e o robô amarelo é o dono, já na imagem
da esquerda a bola está em movimento, aproximadamente na direção do robô azul,
que nesse caso é o dono da bola.


\begin{figure}[h]
  \centering
  \includegraphics[width=0.8\linewidth]{gui_pass}
  \caption{Representação de ação de passe}\label{fig:gui_pass}
\end{figure}

\begin{figure}[h]
  \centering
  \includegraphics[width=0.8\linewidth]{gui_kick}
  \caption{Representação de ação de chute}\label{fig:gui_kick}
\end{figure}

\begin{figure}[h]
  \centering
  \includegraphics[width=0.8\linewidth]{gui_move}
  \caption{Representação das ações de movimentação}\label{fig:gui_move}
\end{figure}

As decisões tomadas possuem representação gráfica, desenhando individualmente
cada ação.  As ações de passe podem ser vistas na Figura~\ref{fig:gui_pass} uma
linha tracejada da bola para o receptor do passe, nessa mesma figura está
exemplificado a representação dos robôs que podem receber passe.

Similarmente a Figura~\ref{fig:gui_kick} apresenta a representação da ação de
chute.  E por fim, o último tipo de ação e mais comum, a de movimentação, pode
ser vista na Figura~\ref{fig:gui_move}.

\FloatBarrier

% vim: tw=80 et ts=2 sw=2 sts=2 ft=tex spelllang=pt_br,en


\section{Influência dos Parâmetros no Comportamento do Time}

Nesta secção são apresentados os diferentes comportamentos que podem ser obtidos
através da modificação dos parâmetros apresentados no
Capítulo~\ref{cap:modelagem}.  Cada parâmetro é modificado e os resultados na
mudança do planejamento são evidenciados.

Os valores iniciais dos parâmetros modificados nos experimentos realizados são:

% TODO: Substituir nomes das variáveis, conforme a modelagem
\begin{enumerate}
  \item Distância de movimentação da bola: 15cm
  \item Abertura mínima para chute a gol: 18.0 graus
%  \item $p_{dist{\ }max{\ }mover} = 0$
%  \item $p_{dist{\ }total{\ }mover} = 0$
%  \item $p_{ataque} = 1000$
  \item $p_{abertura{\ }gol_{ad}} = 10$
  \item $p_{abertura{\ }gol_{c}} = 180$
\end{enumerate}

As Figuras~\ref{fig:default_atq}~e~\ref{fig:default_def} apresentam o
planejamento em um ambiente de ataque e defesa, respectivamente, com os
parâmetros iniciais apresentados anteriormente.

\begin{figure}[H]
  \centering
  \includegraphics[width= 0.8\linewidth]{result/default_atq}
  \caption{Planejamento com os parâmetros iniciais no
           ataque}\label{fig:default_atq}
\end{figure}
\begin{figure}[H]
  \centering
  \includegraphics[width= 0.8\linewidth]{result/default_def}
  \caption{Planejamento com os parâmetros iniciais
           na defesa}\label{fig:default_def}
\end{figure}

\subsection{Correção do Gap devido a movimentação da bola}
% TODO

\begin{figure}[H]
  \centering
  \includegraphics[width= 0.8\linewidth]{result/default_atq}
  \includegraphics[width= 0.8\linewidth]{result/default_def}
  \caption{Planejamento com os parâmetros iniciais no
           ataque (acima) e na defesa (abaixo)}\label{fig:default}
\end{figure}

%\subsection{Distância Total dos Moves}
Somente o peso do custo da distância total dos moves foi
alterado para $10$. Os resultados no planejamento são
apresentados na figura~\ref{fig:mov_dist_total_10}.

\begin{figure}[H]
  \centering
  \includegraphics[width= 0.8\linewidth]{result/mov_dist_total_atq_10}
  \includegraphics[width= 0.8\linewidth]{result/mov_dist_total_def_10}
  \caption{Planejamento com os parâmetros iniciais e o peso do
           custo da distância total dos moves alterado para $10$.
           Ataque (acima) e na defesa (abaixo)}\label{fig:mov_dist_total_10}
\end{figure}

%\subsection{Distância Máxima dos Moves} 
Somente o peso do custo da distância máxima dos moves foi
alterado para $10$. Os resultados no planejamento são
apresentados na figura~\ref{fig:mov_dist_max_10}.

\begin{figure}[H]
  \centering
  \includegraphics[width= 0.8\linewidth]{result/mov_dist_max_atq_10}
  \includegraphics[width= 0.8\linewidth]{result/mov_dist_max_def_10}
  \caption{Planejamento com os parâmetros iniciais e o peso do
           custo da distância máxima dos moves alterado para $10$.
           Ataque (acima) e na defesa (abaixo)}\label{fig:mov_dist_max_10}
\end{figure}

%\subsection{Custo do Ataque}

Somente o peso do custo do ataque foi alterado para $5000$.  Os resultados no
planejamento são apresentados na Figura~\ref{fig:atack_5000}.

\begin{figure}[H]
  \centering
  \includegraphics[width= 0.8\linewidth]{result/atack_atq_5000}
  \includegraphics[width= 0.8\linewidth]{result/atack_def_5000}
  \caption{Planejamento com os parâmetros iniciais e o peso do custo do ataque
  alterado para $5000$.  No ataque (acima) e na defesa (abaixo)}\label{fig:atack_5000}
\end{figure}

% vim: tw=80 et ts=2 sw=2 sts=2 ft=tex spelllang=pt_br,en

\subsection{Custo das Aberturas vistas por $r \in T_c$}

Somente o peso do custo das aberturas do gol vistas pelos robôs do time foi
alterado para $1000$. Os resultados no planejamento são apresentados na
Figura~\ref{fig:see_enemy_goal_1000}. Fica evidente em ambos os ambientes de
ataque e defesa que os robôs se posicionaram o mais próximo possível do gol de
$T_{ad}$, se limitando apenas pela penalização por proximidade do gol
adversário.

\begin{figure}[H]
  \centering
  \includegraphics[width= 0.8\linewidth]{result/see_enemy_goal_atq_1000}
  \includegraphics[width= 0.8\linewidth]{result/see_enemy_goal_def_1000}
  \caption{Planejamento com os parâmetros iniciais e com o peso do custo das
  aberturas do gol vistas pelos robôs do time igual a $1000$.  No ataque (acima)
  e na defesa (abaixo)}\label{fig:see_enemy_goal_1000}
\end{figure}

% vim: tw=80 et ts=2 sw=2 sts=2 ft=tex spelllang=pt_br,en

\subsection{Custo das Aberturas vistas por $r \in T_{ad}$}

Somente o peso do custo das aberturas do goal vistas pelos robôs do time
adversário alterado para zero. Os resultados no planejamento são apresentados na
Figura~\ref{fig:block_goal_0}. Houve uma redução nas posições de defesa com e
sem posse de bola.

\begin{figure}[H]
  \centering
  \includegraphics[width= 0.8\linewidth]{result/block_goal_atq_0}
  \includegraphics[width= 0.8\linewidth]{result/block_goal_def_0}
  \caption{Planejamento com os parâmetros iniciais e com o custo das aberturas
  do goal vistos pelos robôs do time nulo.  No ataque (acima) e na defesa (abaixo)}\label{fig:block_goal_0}
\end{figure}

% vim: tw=80 et ts=2 sw=2 sts=2 ft=tex spelllang=pt_br,en

\subsection{Gap mínimo para chute à gol}
Somente o peso do parâmetro que permite chute à gol
foi alterado para $5$. Os resultados no planejamento são
apresentados na figura~\ref{fig:min_gap_5}.

\begin{figure}[H]
  \centering
  \includegraphics[width= 0.8\linewidth]{result/min_gap_def_5}
  \includegraphics[width= 0.8\linewidth]{result/min_gap_atq_5}
  \caption{Planejamento com os parâmetros iniciais e com o gap
           mínimo para chute alterado para $5$.
           No ataque (acima) e na defesa (abaixo)}\label{fig:min_gap_5}
\end{figure}


\section{Goleiro}

Conforme pode ser visto nos experimentos anteriores, somente em um experimento o
goleiro não surgiu.  A penalização foi desconsiderada para um único robô de
$T_c$.  Isso é interessante, pois não foi necessário restringir o movimento
deste robô, permitindo que este se mova conforme a valoração de $f_U$.

% TODO: More games to include this section. Only two games are not enough.
%\section{Desempenho em Jogo}
%Foram realizados dois jogos supervisionados de dez minutos cada.
%O time $T_1$ jogou com os parâmetros iniciais apresentados no
%início deste capítulo. Já no time $T_2$ foi modificado
%somente o peso $p_{ataque}$ de 1000 para 3530. Os resultados são apresentados
%na Tabela~\ref{tab:games}.
%
%\begin{table}[H]
%  \begin{center}
%  \begin{tabular}{|c|c|}
%    \hline
%    Times      & Resultado \\
%    \hline
%    $T_1$ vs $T_1$ &  0 vs 0   \\
%    \hline
%    $T_1$ vs $T_2$ &  0 vs 1   \\
%    \hline
%  \end{tabular}
%  \caption{Resultado dos jogos $T_1$ contra $T_1$ e $T_1$ contra $T_2$}\label{tab:games}
%  \end{center}
%\end{table}
%
%Conforme observado, houve um equilibrio durante a
%
%% OBS:                  |  It seams like he lost
%%                       |  his mind here...
%%                      \ / more than 10 scretches
%%                       V  only in the subscription number...
%
%partida entre $T_1$ e $T_1$. Com uma pequena
%modificação, pode-se ter um desequilíbrio na
%partida.

% vim: tw=80 et ts=2 sw=2 sts=2 ft=tex spelllang=pt_br,en
