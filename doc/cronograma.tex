\chapter{Cronograma}\label{cap:cronograma}

%Conforme Tabela~\ref{tab:crono}.

\begin{table}[h]
  \begin{center}
    \begin{tabular}{|c|c|c|c|c|c|c|c|c|c|}
      \hline
                        &      &      &      &      &      &      &      &      \\
          Data          & Ago  & Set  & 0ut  & Nov  & Fev  & Mar  & Abr  & Mai  \\
                        &      &      &      &      &      &      &      &      \\
      \hline
      Modelagem do      &  ok  &  ok  &      &      &      &      &      &      \\
      mapeamento        &      &      &      &      &      &      &      &      \\
      \hline
      Prova de          &      &      &  ok  &  ok  &      &      &      &      \\
      conceito          &      &      &      &      &      &      &      &      \\
      \hline
      Implementação     &      &      &  ok  &  ok  &      &      &      &      \\
      da arquitetura    &      &      &      &      &      &      &      &      \\
      \hline
      Implementação     &      &      &      &      &  x   &  x   &  x   &      \\
      do algoritmo      &      &      &      &      &      &      &      &      \\
      \hline
      Otimização do     &      &      &      &      &      &      &      &   x  \\
      algoritmo (opcional)&    &      &      &      &      &      &      &      \\
      \hline
    \end{tabular}
    \caption{Cronograma}
    \label{tab:crono}
  \end{center}
\end{table}


Estado atual: implementação do algoritmo em progresso.

Legenda:

\begin{itemize}
  \item Modelagem do mapeamento: discutido no capítulo~\ref{cap:mapeamento};
  \item Prova de conceito: implementar execução, pelo menos no simulador,
    do planejamento no formato que o algoritmo deve retornar.
  \item Implementação da arquitetura: implementar todo o programa incluindo um
    \textit{stub} do algoritimo do minimax, isto é, o programa final a menos da
    corretude do algoritmo.
  \item Implementação do algoritimo: avançar o algoritmo ao ponto que a solução
    retornada esteja correta e prática, isto é pode ser utilizada para uma
    partida, ainda que somente no simulador.
  \item Otimização do algoritmo (opcional): melhorar a solução para alcançar uma
    taxa ou profundidade maior, como por exemplo implementando a prunagem
    $alpha-beta$.
\end{itemize}

% vim: tw=80 et ts=2 sw=2 sts=2
