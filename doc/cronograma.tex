\chapter{Cronograma}\label{cap:cronograma}

%Conforme Tabela~\ref{tab:crono}.
%\begin{table}[Ht!]
%   \begin{center}
%     \begin{tabular}{|c|c|c|c|c|c|c|c|c|c|}
%       \hline
%                         &      &     &     &     &     &     &     &         \\
%           Data          &  Ago & Set & 0ut & Nov & Fev & Mar & Abr & Mai      \\
%                         &      &     &     &     &     &     &     &         \\
%       \hline                                                                 
%       Modelagem do      &   x  &  x  &     &     &     &     &     &         \\
%       mapeamento        &      &     &     &     &     &     &     &         \\
%       \hline                                                                 
%       Prova de          &      &     &  x  &  x  &     &     &     &         \\
%       conceito          &      &     &     &     &     &     &     &         \\
%       \hline                                                                 
%       Implementação     &      &     &  x  &  x  &     &     &     &         \\
%       da arquitetura    &      &     &     &     &     &     &     &         \\
%       \hline                                                                 
%       Implementação     &      &     &     &     &  x  &  x  &  x  &         \\
%       do algoritmo      &      &     &     &     &     &     &     &         \\
%       \hline                                                                 
%       Otimização do     &      &     &     &     &     &     &     &   x     \\
%       algoritmo(opcional)&      &     &     &     &     &     &     &         \\
%       \hline
%     \end{tabular}
%   \caption{Cronograma}
%   \label{tab:crono}
%   \end{center}
% \end{table}

Para facilitar a compreensão do projeto, optou-se por dividir o projeto em módulos
que seguirão o cronograma abaixo.

\begin{table}[ht]
\resizebox{\textwidth}{!}{
\begin{tabular}{|c|c|c|c|c|c|c|c|c|c|c|c|c|c|c|c|}
\hline
 Ano   & \multicolumn{5}{|c|}{2014} & \multicolumn{5}{|c|}{2015}\\
\hline
\rowcolor[gray]{.8} Tarefas & Ago & Set & Out & Nov & Dez & Jan & Fev & Mar & Abr & Mai \\
\hline
\hline
% ----  & Ago   & Set   & Out    & Nov    & Dez   & Jan   & Fev  & Mar  & Abr  & Mai  \\
 MMP    & \azck & \azck & \azck  & \azl   & \azl  & \azl  & \azl & \azl & \azl &  ·   \\
\hline 
 PVC    &  ·    & ·     & \vdck  & \vdck  & \vrd  & \vrd  & \vrd & \vrd & \vrd &  ·   \\
\hline 
 IMPARQ & ·     & ·     & \vduck & \vduck & \vdu  & \vdu  & \vdu & \vdu & \vdu &  ·   \\
\hline
 IMPALG & ·     & ·     & ·      & ·      & \vdd  & \vdd  & \vdd & \vdd & \vdd &  ·   \\
\hline 
 TST    &  ·    & ·     & ·      & ·      & ·     & ·     & \mr  & \mr  & \mr  &  ·   \\
\hline 
 RED    & \vmck & \vrm  & \vmck  & \vmck  & \vrm  & \vrm  & \vrm & \vrm & \vrm &  ·   \\
\hline 
 DEF    &  ·    & ·     & ·      & ·      & ·     & ·     & ·    & ·    & ·    & \ovd \\
\hline
\end{tabular}
}
\caption{Cronograma --- {\footnotesize MMP: Modelagem do mapeamento; PVC: Prova de conceito;
IMPARQ: Implementação da arquitetura; IMPALG: Implementação do Algoritmo; RED: Redação do
texto; TST: Testes; DEF: Defesa}\label{tab:crono}}
\end{table}

\begin{itemize}
  \item Estado atual: implementação do algoritmo em progresso.
  \item Legenda:
    \begin{itemize}
      \item Modelagem do mapeamento: discutido no capítulo~\ref{cap:mapeamento};
      \item Prova de conceito: implementar execução, pelo menos no simulador, do planejamento
            no formato que o algoritmo deve retornar.
      \item Implementação da arquitetura: implementar todo o programa incluindo um stub do
            algoritimo do minimax, isto é, o programa final a menos da corretude do algoritmo.
      \item Implementação do algoritimo: avançar o algoritmo ao ponto que a solução retornada
            esteja correta e prática, isto é, pode ser utilizada para uma partida real ou
            simulada.
      \item Testes: realização de testes em partidas e coleta de dados obtidos através das
            métricas estabelecidas na seção~\ref{sec:metricas}.
      \item Redação: Desenvolvimento da monografia.
      \item Defesa: Apresentação final do projeto em razão da VF de PFC.
  \end{itemize}
\end{itemize}
% vim: ts=2 sw=2 sts=2
