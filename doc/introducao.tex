\chapter{Introdução}

Cada vez mais a robótica tem feito parte da sociedade atual. Um dos problemas
atuas da robótica é o planejamento em ambientes multiagentes dinâmicos, em
que ao menos um agente não é controlado. Um bom exemplo de um desses ambientes
dinâmicos é um jogo de futebol de robôs, onde um grupo de robôs é controlado por
uma IA e o outro grupo por outra IA independente.

Devido a alta complexidade desses ambientes, não é viável o planejamento considerando
diretamente as leis físicas. Devido a isso, limita-se as ações possíveis do robô
no modelo utilizado no planejamento para que se possa simular mais situações em tempo
hábil, uma vez que o ambiente esta continuamente sujeito a modificações. Entretanto,
para que as simulações sejam válidas, o robô real deve estar em sintonia com seu
modelo. Com efeito, o robô real deve executar os comandos conforme o robô simulado,
caso o mesmo ambiente simulado seja encontrado na prática.

A ideia de robôs jogando futebol foi mencionada pela primeira vez pelo professor
Alan Mackworth (University of British Columbia, Canadá) em um artigo intitulado "On
Seeing Robots", apresentado no Vision Interface 92 e posteriormente publicado em um
livro chamado Computer Vision: System, Theory and Applications. Independentemente,
um grupo de pesquisadores japoneses organizou um Workshop no Ground Challenge in
Artificial Intelligence, em Outubro de 1992, Tóquio, discutindo e propondo problemas que
representavam grandes desafios. Esse Workshop os levou a sérias discussões sobre usar
um jogo de futebol para promover ciência e tecnologia. Estudos foram feitos para analisar
a viabilidade dessa ideia. Os resultados desses estudos mostram que a ideia era viável,
desejável e englobava diversas aplicações práticas. Em 1993, um grupo de pesquisadores,
incluindo Minoru Asada, Yasuo Kuniyoshi e Hiroaki Kitano, lançaram uma competição
de robótica chamada de Robot J-League (fazendo uma analogia à J-League, nome da Liga
Japonesa de Futebol Profissional). Em um mês, vários pesquisadores já se pronunciavam
dizendo que a iniciativa deveria ser estendida ao âmbito internacional. Surgia então, a
Robot World Cup Initiative (RoboCup).

RoboCup é uma competição destinada a desenvolver os estudos na área de robótica
e Inteligência Artificial (IA) por meio de uma competição amigável. Além disso, ela tem
como objetivo, até 2050, desenvolver uma equipe de robôs humanoides totalmente autô-
nomos capazes de derrotar a equipe campeã mundial de futebol humano. A competição
possui várias modalidades. Neste trabalho, será analisada a Small Size Robot League
(SSL), também conhecida como F180. De acordo com as regras da SSL, as equipes devem
ser compostas por 6 robôs, sendo um deles o goleiro, que deve ser designado antes do
início do jogo. Durante o jogo, nenhuma interferência humana é permitida com o sistema
de controle dos robôs. É fornecido aos times um sistema de visão global e esses controlam
seus robôs com máquinas próprias. O sistema de controle dos robôs geralmente é externo
e recebe os dados de um conjunto de duas câmeras localizadas acima do campo. Esse sis-
tema de controle processa os dados, determina qual comando deve ser executado por cada
robô e envia este comando através de ondas de rádio aos robôs. Embora seja permitido
que as equipes utilizem sistemas próprios de visão, a maioria das equipes utiliza a visão
centralizada. A figura 1 mostra uma imagem da SSL Robocup 2013, da qual a RoboIME
(Equipe de Futebol de Robôs do Laboratório de Robótica do IME) participou.

\section{Motivação}

O futebol de robôs, problema padrão de investigação internacional, reúne grande parte
dos desafios presentes em problemas do mundo real a serem resolvidos em tempo real. As
soluções encontradas para o futebol de robôs podem ser estendidas, possibilitando o uso
da robótica em locais de difícil acesso para humanos, ambientes insalubres e situações de
risco de vida iminente.
Há diversas novas áreas de aplicação da robótica, tais como exploração espacial e
submarina, navegação em ambientes inóspitos e perigosos, serviço de assistência médica
e cirúrgica, além do setor de entretenimento. Essas áreas podem ser beneficiadas com o
desenvolvimento de sistemas multi robôs. Nestes domínios de aplicação, sistemas de multi
robôs deparam-se sempre com tarefas muito difíceis de serem efetuadas por um único robô.
Um time de robôs pode prover redundância e contribuir cooperativamente para resolver
o problema em questão. Com efeito, eles podem resolver o problema de maneira mais
confiável, mais rápida e mais econômica, quando comparado com o desempenho de um
único robô.

Devido a alta complexidade de sistemas multiagentes dinâmicos, torna-se necessário
um modelo simplificado para que sejam executadas o maior número de simulações possível.
Caso seja possível uma discretização, ter-se-á um número finito de casos para serem
avaliados. Com isso, pode-se aplicar algoritmos como o minimax (aplicados a jogos de
competitivos de soma zero) para encontrar soluções ao problema.

Isso é muito mais desejável que um modelo heurístico de IA, onde as soluções são
criadas com base nos ambientes identificados pelos modeladores. Isso, pois
a modelagem heurística limita o número de jogadas que se pode executar e limita
a capacidade que o computador tem de testar um grande número de possibilidades.

\section{Objetivo}
Implementação de uma arquitetura de controle com um modelo discreto e sequencial da
RoboCup Small Size League.

\section{Justificativa}
Uma arquitetura de controle que simule os diversos ambientes dinamicamente de maneira sequencial
de um ambiente multiagente permite que várias jogadas sejam criadas dinamicamente, diferentemente
de uma arquitetura estática baseada em uma heuristicamente. Com tal mecanismo é possível melhorar
a IA em uso pela RoboIME para tomar decisões que levem a resultados melhores e, como consequência,
ganhar mais partidas. Nenhuma equipe atualmente esta seguindo esta abordagem, mas
os autores acreditam que essa é uma linha de pesquisa promissora.
