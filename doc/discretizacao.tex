\section{Discretização do jogo}\label{sec:mapeamento}

% TODO[improvement]: add pictures for better understanding
% TODO: Add concept of having the ball
Uma das dificuldades de se discretizar um sistema é a necessidade de criar uma
abstração válida para o jogo, de modo que o que ocorra na simulação aconteça na
prática caso a mesma situação simulada seja observada no mundo físico.

Essa modelagem pode ser separada em duas etapas:

\begin{itemize}
  \item Representação do jogo
  \item Execução do planejamento
\end{itemize}

Essas etapas são descritas a seguir.

\subsection{Representação do jogo}

Cada um dos robôs em campo será modelado com as seguintes ações possíveis:
\begin{itemize}
  \item Time com a bola
  \begin{itemize}
    \item robôs sem a bola
    \begin{itemize}
       \item Move
    \end{itemize}
    \item robô com a bola
    \begin{itemize}
       \item Move
       \item Chute
       \item Passe
    \end{itemize}
  \end{itemize}
  \item Time sem a bola:
  \begin{itemize}
    \item Move
  \end{itemize}
\end{itemize}

O nível de complexidade das ações possíveis influi diretamente no número de
ações que poderão ser consideradas a tempo de serem úteis para o jogo real. Por
exemplo, caso não fosse considerada a ação de passe, esta ação ainda sim poderia
acontecer na prática pela composição de outras ações. Entretanto, seriam
necessários mais níveis de
planejamento, uma vez que ela seria a composição de chutes e movimentações. Isso tem a
contrapartida de reduzir o número de estados que podem ser simulados, uma vez
que o tabuleiro é dinâmico e o jogo real ser simultâneo, e não sequencial.
Isso fica mais evidente se fossem utilizadas somente as \textit{skills} para o
planejamento. A principal desvantagem disso é que o planejador teria que
considerar aspectos como colisões e a orientação dos robôs no planejamento
final. Além de ser ineficiente, coisas como posicionamento global dos robôs no
campo não teriam estados suficientes na árvore do jogo para serem úteis.

% TODO: Add reference
Outra questão que se deve ter em mente ao se modelar as ações básicas dos robôs
é definir ações muito complexas. Passando para a linguagem da arquitetura STP
(\textit{Skill, Tactic Play}), as \textit{plays}, e não \textit{tactics}, o
espaço de jogadas seria muito limitado se fossem utilizadas táticas muito
complexas.

Como a complexidade cresce exponencialmente, onde o número de ações básicas é
a base, isso não é um problema que pode ser tratado simplesmente com o aumento
da velocidade de processamento. Deve-se ajustar o nível de abstração de acordo
com os resultados obtidos nos teste práticos.

\subsection{Execução do planejamento}
% TODO[improvement]: specify add images

Esta etapa do modelo é responsável por converter o resultado do planejamento em
comandos mais concretos. Conforme evidenciado na seção anterior, é nesta parte
que o planejamento de trajetória deve ser levado em consideração. Esta parte que
leva em consideração o modelo dinâmico do robô.
Como isso é um problema complexo, com o objetivo de focar o escopo da pesquisa
no planejamento de alto nível, será utilizada a arquitetura de controle do
pyroboime. Essa parte do sistema será detalhada no capítulo~\ref{cap:arch_sys}.

% vim: tw=80 et ts=2 sw=2 sts=2
