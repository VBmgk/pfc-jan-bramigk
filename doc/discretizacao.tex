\section{Discretização do jogo}\label{sec:mapeamento}

% TODO[improvement]: add pictures for better understanding
Uma das dificuldades de se discretizar um sistema é a necessidade de criar uma
abstração válida para o jogo, de modo que o que ocorra na simulação aconteça na
prática caso a mesma situação simulada seja observada no mundo físico.

Essa abstração pode ser separada em duas etapas:

\begin{itemize}
  \item Representação do jogo
  \item Execução do planejamento
\end{itemize}

Essas etapas são descritas a seguir.

\subsection{Representação do jogo}\label{subsec:repres_jogo}

\subsubsection{Posse de bola}

Para simplificar o modelo, foi introduzido o conceito de posse
de bola. O robô que possue a bola é aquele que consegue intercepta-la
no menor tempo possível. Como introduzir um modelo que considere
a dinâmica do robô iria introduzir um custo computacional muito alto,
foi utilizado um modelo simplificado. As simplificações são as
seguintes (somente para o calculo do tempo da interceptação):

\begin{itemize}
  \item A bola se move a velocidade constante
  \item Os robôs conseguem se mover em uma velocidade máxima
        $\hat{r}.vel_{max}$ em qualquer direção a qualquer momento
\end{itemize}

Dadas essas simplificações, o tempo para atingir o ponto de
encontro pode ser calculado da seguinte maneira: 
%  vb.t + pb = vr.t + pr
%  => 0 = (vr^2 - vb^2)t^2 - |pr - pb|^2 - 2.vb.(pb - pr).t
%  => delta = 4.[ (vb.(pb - pr))^2 + (vr^2 - vb^2).|pr - pb|^2]
%  b = -2.vb.(pb - pr) = 2.vb.(pr - pb)
%  b_div_2 = vb.(pr - pb)
%  a = (vr^2 - vb^2)
%  c = -|pr - pb|^2
%  t = -b_div_2 +- sqrt(delta_div_4)
%     -----------------
%            a

\begin{gather}
  \hat{b}.vel \cdot t_{encontro} + \hat{b}.pos = \hat{r}.vel_{max} \cdot t_{encontro} + \hat{r}.pos
\end{gather}

Trabalhando a equação acima, tem-se a seguinte equação do segundo grau:
\begin{gather}  
  0 = (\hat{r}.vel^2_{max}-\hat{b}.vel^2) \cdot t^2 - \parallel \hat{r}.pos - \hat{b}.pos \parallel ^2
     - 2 \cdot \hat{b}.vel \cdot (\hat{b}.pos - \hat{r}.pos) \cdot t_{encontro}
\end{gather}
Logo, tem-se:
\begin{gather}  
  \boxed{t_{encontro} = \frac{ - \hat{b}.vel \cdot (\hat{r}.pos - \hat{b}.pos) \pm \sqrt {\frac{\Delta}{4}}}
                 {(\hat{r}.vel^2_{max} - \hat{b}.vel^2)}}
\end{gather}
Onde $\Delta = 4 \cdot [ ( \hat{b}.vel \cdot (\hat{b}.pos - \hat{r}.pos)) ^2 +
           (\hat{r}.vel_{max}^2 - \hat{b}.vel^2) \cdot \parallel \hat{r}.pos - \hat{b}.pos \parallel ^2]$.

\subsubsection{Ações Consideradas}
% TODO: Citar imagem \ref{fig:robo}
Cada um dos robôs em campo será modelado com as seguintes ações possíveis:
\begin{itemize}
  \item Time no ataque (i.e., com a bola):\\
        $A_{atq} = \lbrace Mover(r_i), Chutar(r_{com{\ }bola}), Passar(r_j,r_{com{\ }bola}):
                    r_i, r_j, r_{com{\ }a{\ }bola} \in Rob_{atq}\rbrace$

  \item Time na defesa (i.e., sem a bola):
        $A_{def} = \lbrace Mover(r_i): r_i \in Rob_{def}\rbrace$
\end{itemize}

% TODO: referenciar ações plan kick and pass do capítulo resultados 
%A Figura~\ref{fig:plan_pass} mostra um planejamento com ações de mover e
%passar. Já a Figura~\ref{fig:plan_kick} mostra um planejamento com uma
%ação de chutar no lugar de passar.

Note que o robô com a bola pode se mover. Entretanto isso é indesejável, já que
a velocidade do passe é maior que a movimentação do robô. Também existe uma
restrição quanto à distância máxima que se pode conduzir a bola que, se não
for respeitada, resulta em penalidade para o time do robô infrator.

O nível de complexidade das ações possíveis influi diretamente no número de
ações que poderão ser consideradas a tempo de serem úteis para o jogo real. Por
exemplo, caso não fosse considerada a ação de passe, esta ação ainda sim poderia
acontecer na prática pela composição de outras ações. Entretanto, seriam
necessários mais níveis de planejamento, uma vez que ela seria a composição de
chutes e movimentações. Isso tem a contrapartida de reduzir o número de estados
que podem ser simulados, uma vez que o tabuleiro é dinâmico e o jogo real ser
simultâneo, e não sequencial. Isso fica mais evidente se fossem utilizadas
somente as \textit{skills} para o planejamento. A principal desvantagem disso
é que o planejador teria que considerar aspectos como colisões e a orientação
dos robôs no planejamento final. Além de ser ineficiente, coisas como
posicionamento global dos robôs no campo não teriam estados suficientes na
árvore do jogo para serem úteis.

Outra questão que se deve ter em mente ao se modelar as ações básicas dos robôs
é definir ações muito complexas. Passando para a linguagem da arquitetura STP
(\textit{Skill, Tactic Play}), se fossem usadas \textit{plays} no lugar de
\textit{tactics}, o espaço de jogadas seria muito limitado e poucas jogadas
seriam geradas pelo computador. Mais detalhes da arquitetura STP podem ser
encontrados em \cite{browning2004stp}.

Como a complexidade cresce exponencialmente, onde o número de ações básicas é
a base, isso não é um problema que pode ser tratado simplesmente com o aumento
da velocidade de processamento. Deve-se ajustar o nível de abstração de acordo
com os resultados obtidos nos teste práticos.

\subsection{Execução do planejamento}
% TODO[improvement]: specify add images

Esta etapa do modelo é responsável por converter o resultado do planejamento em
comandos mais concretos. Conforme evidenciado na seção anterior, é nesta parte
que o planejamento de trajetória deve ser levado em consideração. Esta parte que
leva em consideração o modelo dinâmico do robô.
Como isso é um problema complexo, com o objetivo de focar o escopo da pesquisa
no planejamento de alto nível, será utilizada a arquitetura de controle do
pyroboime. Essa parte do sistema será detalhada no Capítulo~\ref{cap:arquitetura}.

% vim: tw=80 et ts=2 sw=2 sts=2
