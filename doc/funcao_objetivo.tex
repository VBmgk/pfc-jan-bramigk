\section{Função Objetivo}

A função objetivo $f_U$ é definida pela seguinte expressão:

\begin{dmath}
  f_U = \sum c_i 
\end{dmath}

Onde $\lbrace c_i \rbrace$ é o conjunto formado por custos. Os pesos de cada
custo serão denotados por $p_{descrição}$. São eles:
% TODO: Eu tirei a penalização, mas talvez seja melhor explicar
\begin{enumerate}
  \item Custo da Abertura do Gol
  \item Correção da Abertura do Gol Devido a Movimentação da Bola
  \item Distância Total das Ações $Mover(r)$
  \item Distância Máxima das Ações $Mover(r)$
  \item Mudança do Planejamento da Move Table
  \item Custo do Ataque
  \item Custo das Aberturas vistas por $r \in T_c$
  \item Custo das Aberturas vistas por $r \in T_{ad}$
  \item Custo da Defesa
  \item Número de Receptores
  \item Penalização por Proximidade do Gol do Adversário
  \item Penalização por Proximidade
\end{enumerate}

Além dos parâmetros listados acima, outros parâmetros afetam o comportamento do
time:
\begin{enumerate}
  \item Abertura do gol mínima para chute a gol
  \item Número de ramificações
\end{enumerate}

Esses parâmetros foram os primeiros utilizados no programa.  Ao longo dos testes
realizados, foram criados novos parâmetros.  Entretanto, os parâmetros listados
são suficientes para ilustrar a mudança no comportamento do time.

% OBS: o Maj Duarte circulou o título, mas não sei se foi
%      pelos dois erros de português que tinha aqui...
\subsection{Custo da Abertura do Gol}\label{subsec:custo_gap}

O objetivo deste parâmetro é valorizar as aberturas maiores de acordo com a soma
total das aberturas com relação a cada $r \in T$ e com aberturas maiores.  O
cálculo deste custo é feito da seguinte maneira:

\begin{multline} 
 c_{abertura{\ }gol} = taxa_{abertura{\ }total/max{\ }gol} .
   \sum_{r_i \in T} abertura{\ }gol(bola, r)\\
   + (1 - taxa_{abertura{\ }total/max{\ }gol}) .
   max \lbrace abertura{\ }gol(bola, r): r \in T \rbrace
\end{multline}

As aberturas são calculadas de acordo com a posição de um determinado robô $r$
(obstáculo) e a posição atual da bola. Isso gera um conjunto de segmentos de
reta $\lbrace abertura{\ }gol(bola, r): r \in T \rbrace$.

Este custo afeta diretamente os seguintes custos:
\begin{itemize}
  \item Custo do Ataque
  \item Custo da Defesa
  \item Custo das aberturas do gol vistas por $r\in T_c$
  \item Custo das aberturas do gol vistas por $r\in T_{ad}$
\end{itemize}

\subsection{Correção da Abertura do Gol Devido a Movimentação da Bola}

O objetivo deste parâmetro é incorporar a movimentação da bola pelo atacante no
cálculo da abertura do gol. Isso é importante, pois defensores mais próximos são
mais facilmente driblados que defensores mais distantes. Isso pode ser
visualizado na figura \ref{fig:kick_pos}, onde foi considerada uma variação de
15cm e de 0cm. Essa variação é medida na linha normal à reta formada pela bola e
pelo robô em questão.

% TODO: Adicionar gráfico detalhado as contas
%       vide arquivo board.cpp para detalhes de implementação

\begin{figure}[H]
  \centering
  \includegraphics[height=0.4\linewidth]{kick_pos_var_1}
  \includegraphics[height=0.4\linewidth]{kick_pos_var_2}
  \caption{Abertura do gol considerando-se uma variação de 15cm na posição da
  bola (esquerda) e sem variação (direita)}\label{fig:kick_pos}
\end{figure}


\subsection{Distância Total das Ações $Mover(r)$}

O objetivo deste parâmetro é valorar o custo da mudança de estado na função
objetivo. Ele é formado pela soma das distâncias que cada ação $Mover(r)$ irá
percorrer partindo das respectivas posições do estado atual.

\begin{dmath}
 c_{dist{\ }total{\ }mover} = p_{dist{\ }total{\ }mover} .
 \sum_{r_i \in T_c} \lVert pos_{r_i} - Mover(r_i)\rVert
\end{dmath}

\subsection{Distância Máxima das Ações $Mover(r)$}

O objetivo deste parâmetro é incorporar o custo da mudança de estado na função
objetivo. Ele é formado pela ação $Mover(r)$ do planejamento atual com maior
deslocamento.

\begin{gather}
 c_{dist{\ }max{\ }mover}= p_{dist{\ }max{\ }mover} .
 max \lbrace r_i \in T_c : \lVert pos_{r_i} - Mover(r_i)\rVert \rbrace
\end{gather}

\subsection{Mudança do Planejamento da Tabela de Decisão}\label{subsec:change_cost}

O objetivo deste parâmetro é evitar mudanças grandes no planejamento da ação
$Mover(r)$. Uma das razões para isso é evitar um modelo dinâmico exato neste
nível de planejamento, já que isso aumentaria muito o custo computacional desta
etapa do planejamento e, consequentemente, reduziria o número de simulações
possíveis. Por essas razões mudanças no planejamento das ações $Mover(r)$
inicialmente foram penalizadas de acordo com a distância euclidiana entre o
$Mover_p(r)$ planejado anteriormente e o $Mover_{m}(r)$ modificado, conforme a
equação:

\begin{dmath}
 c_{mudança{\ }mover} = p_{mudança{\ }mover} .
 max \lbrace \sum_{r_i \in T_c} \lVert Mover(r_i) - Mover_p(r_i)\rVert \rbrace
\end{dmath}

Isso permite que sejam selecionados $Mover_{m}(r)$ mais próximos do
$Mover_p(r)$, refinado assim o planejamento anterior. Entretanto, um parâmetro
mais significativo é o valor absoluto do ângulo entre essas ações e a posição de
$r$, $\lVert ang(r, Mover_{m}(r), Mover_p(r)) \rVert$.  Dessa maneira, tem-se
que este custo é computado da seguinte maneira:

\begin{dmath}
 c_{mudança{\ }mover} = p_{mudança{\ }mover} .
 max \lbrace \sum_{r_i \in T_c} \lVert ang(r, Mover_{m}(r), Mover_p(r)) \rVert \rbrace
\end{dmath}

\subsection{Custo do Ataque}

Este custo valoriza situações nas quais o time em questão possui o domínio da
bola (descrito na Subsecção~\ref{subsec:repres_jogo}).

\begin{dmath}
 c_{ataque} = p_{ataque} . abertura{\ }gol_{ad}
\end{dmath}

Onde $abertura{\ }gol_{ad}$ é o valor da abertura do gol adversário visto pelo
robô que tem domínio da bola.

\subsection{Custo da Defesa}

Quando não se tem o domínio da bola a abertura do gol do time em questão visto
pelo robô do time adversário que tem a bola é utilizado como penalização
adicional.

\begin{dmath}
  c_{defesa} = p_{defesa} .
   \sum_{r_i \in T_ad} abertura{\ }gol_c(r_i)
\end{dmath}

\subsection{Custo das Aberturas do Gol Vistas por $r\in T_c$}

Este custo tem o objetivo de valorizar configurações nas quais os robôs que não
tem a bola possuam visada para o gol do time adversário.  Isso é desejável, já
que permite que um gol seja feito caso o robô que esta com a posse de bola
execute um passe para um dos robôs em questão.

\begin{dmath}
   c_{abertura{\ }gol{\ }ad} = p_{abertura{\ }gol_{ad}} .
    \sum_{r_i \in T_c} abertura{\ }gol_{ad}(r_i)
\end{dmath}

\subsection{Custo das Aberturas do Gol Vistas por $r\in T_{ad}$}

Este custo é o análogo do custo anterior, mas para o time adversário. Ele visa
penalizar situações nas quais os robôs adversários tenham grande visada (i.e.,
abertura) para o gol.

\begin{dmath}
   c_{bloqueio{\ }gol} = - p_{bloqueio{\ }gol} .
    \sum_{r_i \in T_{ad}} abertura{\ }gol_{ad}(r_i)
\end{dmath}

\subsection{Número de Receptores}

Este custo visa valorizar situações nas quais existam vários robôs que possam
receber passe.

\begin{dmath}
  c_{receptores} = p_{receptores} .
   num \lbrace r_{receptor_i} \rbrace
\end{dmath}

\subsection{Penalização por Proximidade do Gol do Adversário}

Esta penalização visa evitar que os robôs que estão no ataque se concentrem
dentro da área do time adversário. Caso este parâmetro não seja adicionado, é
esse o comportamento resultante do custo das aberturas vistas pelos robôs do
time em questão.  A partir de uma determinada distância, é adicionada uma
parcela de penalização na função objetivo.

\begin{dmath}
  c_{penalização{\ }prox} = - p_{penalização{\ }prox}
    \sum num \lbrace r_{perto} \rbrace
\end{dmath}

\subsection{Abertura Mínima para Chute a Gol}

Este parâmetro é o ângulo total da abertura do gol para permitir o chute pelo
atacante. Ele é relevante, pois afeta o comportamento do atacante e, como
consequência, o comportamento do time durante a posse de bola. Quando a abertura
do gol é menor que este valor, o atacante tem ação de chutar em direção ao goal.

% TODO: Adicionar imagem para ilustrar este caso, já que não tem
%       equações envolvidas

\subsection{Número da Ramificação}

Este parâmetro é o número de possibilidades que serão simulados em cada iteração
do algorítimo de avaliação. Ele influencia o tempo de resposta do planejamento.

% vim: tw=80 et ts=2 sw=2 sts=2 ft=tex spelllang=pt_br,en
