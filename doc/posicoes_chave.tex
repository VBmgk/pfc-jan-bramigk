\subsection{Posições Chave}\label{subsec:pos_chave}

Devido ao grande número de possibilidades para as posições dos robôs, somente a
geração de posições aleatórias não gera bons resultados para as ações do tipo
$Mover(r_i)$. Portanto, para direcionar a busca, foi desenvolvida uma maneira de
se sugerir posições para as ações em questão com o auxílio da interface gráfica.
A Figura~\ref{fig:pos_chave_barreira} mostra a sugestão de uma barreira.  Já na
Figura~\ref{fig:pos_chave_lateral} são sugeridas posições laterais.

\begin{figure}[H]
  \centering
  \includegraphics[width= 0.8\linewidth]{pos_chave_barreira}
  \caption{Sugestão de uma barreira (em rosa)}\label{fig:pos_chave_barreira}
  \includegraphics[width= 0.8\linewidth]{pos_chave_lateral}
  \caption{Sugestão de posições laterais (em rosa)}\label{fig:pos_chave_lateral}
\end{figure}

Com o objetivo de verificar a utilidade das posições chave, foi criada uma
maneira de identificar se uma posição chave foi utilizada.

As posições chave são utilizadas das seguinte maneira:
\begin{itemize}
  \item Caso um número de posições inferior ao número de robôs sejam sugeridas,
    são criadas ações aleatórias para completar a sugestão;
  \item As posições sugeridas são atribuídas para o robô mais próximo da posição
    sugerida. Isso é feito na ordem em que as posições foram sugeridas.
\end{itemize}

% vim: tw=80 et ts=2 sw=2 sts=2 ft=tex spelllang=pt_br,en
