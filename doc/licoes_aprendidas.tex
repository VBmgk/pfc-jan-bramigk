\chapter{Lições Aprendidas com Implementações Anteriores}

\section{Abordagens}
- Heurística pura
- Otimização
- Baseada em Jogos
- Otimização com modelagem do oponente

\section{Objetivos de longo prazo}
Desenvolver uma inteligência que comece com
uma solução baseada em jogos, que
gradativamente vai aprendendo o
comportamento de seu oponente e vai se
transformando em uma inteligência baseada
em otimização, na medida do possível, e
enquanto for necessário.

\section{Problemas aprendidos}
A implementação de uma abordagem baseada
em jogos com uma simulação física complicada
tornava a resposta lenta, logo a qualidade da
solução obtida em tempo real era inferior
àquela obtida por uma heurística pura.
Discutindo com o Felix, concluímos que o
BKBGT tinha o mesmo tipo de problema, logo
também optou-se por heurísticas puras nos
times omnidirecionais.

\section{Proposta de solução}
Explorar o fato de que no jogo com robôs
omnidirecionais não são necessárias as
trajetórias elaboradas que o primeiro time
usava.
Aproximar a simulação física por um jogo de
tabuleiro. Ou seja, levar a simulação para um
nível mais conceitual.

\section{Minimax para robôs omnidirecionais}
Jogadas possíveis de cada robô:
  - Pegar a bola parada, se o oponente não estiver mais próximo da bola que o
  - nosso robô.
  - Pegar a bola em movimento, se for possível, escolhendo um ponto aleatório
  - da trajetória da bola.
  - Passar a bola, se não tiver ninguém que possa interceptar a bola.
  - Chutar a gol, se não tiver ninguém no caminho da bola.
  - Deslocar-se para outro ponto escolhido aleatoriamente, se não tiver ninguém
  - em seu caminho, ou utilizar a célula de Voronoi.
  - Deslocar-se para o ponto escolhido na última jogada. ( Aprendizagem local )

\section{Problemas}
Infelizmente não conseguimos realizar o
mapeamento do modelo conceitual para os
controles dos robôs. Nunca descobrimos o
problema.
Considere que o time que esteja se defendendo
esteja na raiz da árvore MiniMax. O goleiro deste
time sempre conseguirá ficar entre a bola e o gol.

\section{Ideias}
Talvez seja melhor fazer algum tipo de
desacoplamento do comportamento dos
jogadores para melhorar o aproveitamento da
CPU\@.
Talvez seja melhor definir algumas jogadas de
movimento com finalidade especifica, por
exemplo, bloquear a bola defensivamente. O
problema é o aumento do tempo de
processamento.

\section{Considerações}
O benefício da substituição das jogadas do
oponente por jogadas aprendidas é duplo,
aumentamos a efetividade das nossas jogadas,
e simultaneamente permitimos que o nosso
número de jogadas consideradas possa
aumentar.
Considero o problema de encontrar um modelo
para o comportamento do oponente a parte
mais desafiadora do projeto.
