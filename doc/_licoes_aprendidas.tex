%\chapter{Lições Aprendidas com Implementações Anteriores}
\chapter{Experiência com Implementações Prévias}\label{cap:licoes_aprendidas}
% TODO[improvement]: make it objective

\section{Abordagens}

As abordagens consideradas para criação de uma IA para a SSL são:
\begin{itemize}
 \item Heurística pura
 \item Otimização
 \item Baseada em Jogos
 \item Otimização com modelagem do oponente
\end{itemize}

\section{Objetivos de longo prazo}
Desenvolver uma inteligência que comece com uma solução baseada em jogos, que
gradativamente vai aprendendo o comportamento de seu oponente e vai se
transformando em uma inteligência baseada em otimização, na medida do possível, e
enquanto for necessário.

\section{Problemas aprendidos}
% TODO: citar cap madeira
A implementação de uma abordagem baseada em jogos com uma simulação física complicada
tornava a resposta lenta, logo a qualidade da solução obtida em tempo real era inferior
àquela obtida por uma heurística pura.
Foi tentado um minimax com simulador físico, mas o custo computacional das simulações
associado ao cresciemento exponencial da árvore minimax não permite um controle eficiente
dos robôs em um ambiente dinâmico com agentes que não são controláveis.
Foi tentado um minimax com chaveamento de heurística, mas fracasou devido ao
custo computacional. Logo também optou-se por heurísticas puras nos
times com robôs omnidirecionais.

\section{Proposta de solução}
Explorar o fato de que no jogo com robôs omnidirecionais não são necessárias as
trajetórias elaboradas que o primeiro time usava.
Aproximar a simulação física por um jogo de tabuleiro. Ou seja, levar a simulação para um
nível mais conceitual.

\section{Minimax para robôs omnidirecionais}
Modelagem das jogadas possíveis de cada robô:
\begin{itemize}
 \item Pegar a bola parada, se o oponente não estiver mais próximo da bola que o
 nosso robô.
 \item Pegar a bola em movimento, se for possível, escolhendo um ponto aleatório
 da trajetória da bola.
 \item Passar a bola, se não tiver ninguém que possa interceptar a bola.
 \item Chutar a gol, se não tiver ninguém no caminho da bola.
 \item Deslocar-se para outro ponto escolhido aleatoriamente, se não tiver ninguém
 em seu caminho, ou utilizar a célula de Voronoi.
 \item Deslocar-se para o ponto escolhido na última jogada. ( Aprendizagem local )
\end{itemize}

\section{Problemas}
Não foi realizado um mapeamento do modelo conceitual para os
controles dos robôs. Não foi descoberto o problema.

Considere que o time que esteja se defendendo esteja na raiz da árvore MiniMax.
Neste caso, o goleiro deste time sempre conseguirá ficar entre a bola e o gol.

\section{Ideias}
Provavelmente é necessário fazer algum tipo de desacoplamento do comportamento dos
jogadores para melhorar o aproveitamento da CPU\@.
Talvez seja melhor definir algumas jogadas de movimento com finalidade especifica,
por exemplo, bloquear a bola defensivamente. O problema é o aumento do tempo de
processamento.

\section{Considerações}
O benefício da substituição das jogadas do oponente por jogadas aprendidas é duplo,
aumenta-se a efetividade das jogadas, e simultaneamente possibilita que o
número de jogadas consideradas possa aumentar. O problema de encontrar um
modelo para o comportamento do oponente é a parte mais desafiadora do projeto.

% vim: tw=80 et ts=2 sw=2 sts=2 ft=tex
