\documentclass{ime-beamer}
\usepackage[portuges]{babel}
\usepackage[utf8]{inputenc}
\usepackage{graphicx}
\usepackage[portuguese]{algorithm2e}% Para escrever algorítimos
\usepackage{listings}			% Para usar \lstinputlisting e incluir código
\usepackage{subcaption}			% Para usar \begin{subfigure} e colocar figuras (a) e (b) lado a lado

\title{Heurística Estática para\\
       Times Cooperativos de Robôs}
\author{Victor Bramigk\\
        Jan Segre\\
        Paulo F. F. Rosa (Orientador)}

\begin{document}
  \frame{
    \maketitle
  }

  \frame{
    \frametitle{Roteiro}
    \tableofcontents
  }

\section{Introdução}
\frame{
  \frametitle{Robótica}
  \begin{block}{}
    \begin{figure}[H]
      \centering
%     \includegraphics[width = 0.35\linewidth]
%      {figuras/robotica}
       \caption{\textit{iRobot} utilizado na inspeção da
         Central Nuclear de Fukushima I após o sismo e
         tsunami de 2011 em Sendai}
    \end{figure}
  \end{block}
}
\frame{
  \frametitle{Robótica}
  \begin{block}{}
    \begin{figure}[H]
      \centering
%     \includegraphics[width = 5cm]
%     {figuras/robocup2013}
      \caption{Partida da \textit{Small Size League} (SSL)
      da Robocup 2013}
    \end{figure}
  \end{block}
}
\frame{
  \frametitle{Objetivos}
  \begin{block}{}
    \centering
    Estudar os algoritmos ACO, SA, GA, ANN e lógica fuzzy.
  \end{block}
  \begin{block}{}
    \centering
    Analisar se e como a inteligência artificial de um time de futebol de robôs
    pode ser modelada utilizando esses algoritmos e métodos com as informações
    contidas nos $logs$ (definido a seguir) de um jogo da SSL da RoboCup.
  \end{block}
}
  %\input{sec/problema.tex}
  %\input{sec/metodos.tex}
  %\input{sec/analise.tex}
  %%\addcontentsline{toc}{chapter}{Conclusão}
\chapter{Conclusão}\label{cap:conclusao}

% TODO:
% OBS: - Recontextualizar trabalho
%      - Retificação do cumprimento dos objetivos
%      - Listagem das contribuições
%      - Trabalhos futuros

Este trabalho objetiva desenvolver uma ferramenta de representação
comportamental baseada em otimização para futebol de robôs.  A meta
intermediária é criar um modelo discreto sequencial para o problema do futebol
de robôs. A partir desta discretização, foi desenvolvida uma arquitetura de
controle que seleciona jogadas o mais próximo da jogada ótima possível, de
acordo com uma função de avaliação e dentro do tempo disponível para o
planejamento.

Foi desenvolvido um modelo abstrato do futebol de robôs no
Capítulo~\ref{cap:modelagem}. Este modelo foi base para o programa apresentado
no Capítulo~\ref{cap:arquitetura}.

A ferramenta criada atingiu os objetivos desejados, modificando o comportamento
do time através da modificação dos parâmetros da função objetivo, conforme
mostrado no Capítulo~\ref{cap:resultados}. Também foi desenvolvida com sucesso
uma interface gráfica que permite modificar esses parâmetros em tempo de
execução.

O trabalho atual pode permitir que um time melhor seja criado modificando-se os
parâmetros da função objetivo ou adicionando novos custos à função objetivo.
Através do fornecimento dos parâmetros vencedores da competição anterior, tem-se
um avanço no desempenho do time.

Pode-se melhorar o time utilizando uma maneria automática para realizar as
partidas e avaliar o desempenho do time nessa partida. Isso permitiria que
algorítimos de otimização também fossem utilizados para melhorar os parâmetros.
Para isso, é a necessário um juíz automático para permitir jogos não
supervisionados por humanos.
% TODO: Escolher outra palavra, 'humanos' parece estranho.

% vim: tw=80 et ts=2 sw=2 sts=2 ft=tex

  %\input{sections/referencias.tex}
\end{document}
